\input{../utils/preamble}
\createdgmtitle{12}

\usepackage{tikz}

\usetikzlibrary{arrows,shapes,positioning,shadows,trees}
%--------------------------------------------------------------------------------
\begin{document}
%--------------------------------------------------------------------------------
\begin{frame}[noframenumbering,plain]
%\thispagestyle{empty}
\titlepage
\end{frame}
%=======
\begin{frame}{Recap of previous lecture}
	\begin{block}{Theorem (continuity equation)}
		If $\bff$ is uniformly Lipschitz continuous in $\bx$ and continuous in $t$, then
		\[
			\frac{d \log p_t(\bx(t))}{d t} = - \text{tr} \left( \frac{\partial \bff(\bx(t), t)}{\partial \bx(t)} \right)
		\]
		\vspace{-0.3cm}
		\[
			\log p_1(\bx(1)) = \log p_0(\bx(0)) - \int_{0}^{1} \text{tr}  \left( \frac{\partial \bff(\bx(t), t)}{\partial \bx(t)} \right) dt.
		\]
		\vspace{-0.3cm}
	\end{block}
	\begin{itemize}
		\item \textbf{Discrete-in-time NF}: evaluation of determinant of the Jacobian costs $O(m^3)$ (we need invertible $\bff$).
		\item \textbf{Continuous-in-time NF}: getting the trace of the Jacobian costs $O(m^2)$ (we need smooth $\bff$).
	\end{itemize}
	\begin{block}{Hutchinson's trace estimator}
		\vspace{-0.3cm}
		\[
	  		\log p_1(\bx(1)) = \log p_0(\bx(0)) - \mathbb{E}_{p(\bepsilon)} \int_{0}^{1} \left[ {\color{violet}\bepsilon^T \frac{\partial \bff}{\partial \bx}} \bepsilon \right] dt.
		\]
	\end{block}
	\myfootnotewithlink{https://arxiv.org/abs/1810.01367}{Grathwohl W. et al. FFJORD: Free-form Continuous Dynamics for Scalable Reversible Generative Models, 2018} 
\end{frame}
%=======
\begin{frame}{Recap of previous lecture}
	\begin{block}{Forward pass (Loss function)}
		\vspace{-0.7cm}
		\[
			L(\bx) = - \log p_1(\bx(1) | \btheta) = - \log p_0(\bx(0)) + \int_{0}^{1} \text{tr} \left( \frac{\partial \bff_{\btheta}(\bx(t), t)}{\partial \bx(t)} \right) dt
		\]
	\end{block}
	\vspace{-0.5cm}
	\begin{block}{Adjoint functions}
		\vspace{-0.3cm}
		\[
			\ba_{\bx}(t) = \frac{\partial L}{\partial \bx(t)}; \quad \ba_{\btheta}(t) = \frac{\partial L}{\partial \btheta(t)}.
		\]
		\vspace{-0.5cm}
	\end{block}
	\begin{block}{Theorem (Pontryagin)}
		\vspace{-0.5cm}
		\[
		\frac{d \ba_{\bx}(t)}{dt} = - \ba_{\bx}(t)^T \cdot \frac{\partial \bff_{\btheta}(\bx(t), t)}{\partial \bx}; \quad \frac{d \ba_{\btheta}(t)}{dt} = - \ba_{\bx}(t)^T \cdot \frac{\partial \bff_{\btheta}(\bx(t),  t)}{\partial \btheta}.
		\]
		\vspace{-0.7cm}
		\begin{align*}
			\frac{\partial L}{\partial \btheta(0)} &= \ba_{\btheta}(0) =  - \int_{1}^{0} \ba_{\bx}(t)^T \frac{\partial \bff_{\btheta}(\bx(t), t)}{\partial \btheta(t)} dt + 0\\
			\frac{\partial L}{\partial \bx(0)} &= \ba_{\bx}(0) =  - \int_{1}^{0} \ba_{\bx}(t)^T \frac{\partial \bff_{\btheta}(\bx(t), t)}{\partial \bx(t)} dt + \frac{\partial L}{\partial \bx(1)}
		\end{align*}
	\end{block}
	\myfootnotewithlink{https://arxiv.org/abs/1806.07366}{Chen R. T. Q. et al. Neural Ordinary Differential Equations, 2018}   
\end{frame}
%=======
\begin{frame}{Recap of previous lecture}
	\begin{block}{Forward pass}
		\vspace{-0.3cm}
		\[
		\bx(1) = \bx(0) + \int^{1}_{0} \bff_{\btheta}(\bx(t), t) d t \quad \Rightarrow \quad \text{ODE Solver}
		\]
		\vspace{-0.4cm}
	\end{block}
	\begin{block}{Backward pass}
		\vspace{-0.5cm}
		\begin{equation*}
			\left.
			{\footnotesize 
				\begin{aligned}
					\frac{\partial L}{\partial \btheta(0)} &= \ba_{\btheta}(0) =  - \int_{1}^{0} \ba_{\bx}(t)^T \frac{\partial \bff_{\btheta}(\bx(t), t)}{\partial \btheta(t)} dt + 0 \\
					\frac{\partial L}{\partial \bx(0)} &= \ba_{\bx}(0) =  - \int_{1}^{0} \ba_{\bx}(t)^T \frac{\partial \bff_{\btheta}(\bx(t), t)}{\partial \bx(t)} dt + \frac{\partial L}{\partial \bx(1)} \\
					\bx(0) &= - \int^{1}_{0} \bff_{\btheta}(\bx(t), t) d t  + \bx(1).
				\end{aligned}
			}
			\right\rbrace
			\Rightarrow
			\text{ODE Solver}
		\end{equation*}
		\vspace{-0.4cm} 
	\end{block}
	\textbf{Note:} These scary formulas are the standard backprop in the discrete case.
	\myfootnotewithlink{https://arxiv.org/abs/1806.07366}{Chen R. T. Q. et al. Neural Ordinary Differential Equations, 2018}   
\end{frame}
%=======
\begin{frame}{Outline}
	\tableofcontents
\end{frame}
%=======
\begin{frame}{Summary}
	\begin{itemize}
		\item 
	\end{itemize}
\end{frame}
\end{document} 